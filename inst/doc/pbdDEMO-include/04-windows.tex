

\section[Windows Systems]{Windows Systems}
\label{sec:windows_systems}
\addcontentsline{toc}{section}{\thesection. Windows Systems}

Currently, \pkg{pbdDEMO} supports Windows with
MPICH2 (\url{http://www.mcs.anl.gov/research/projects/mpich2/}).
The binary installations of both MPI systems are available
from the website.
\code{mpich2-1.4.1p1-win-ia32.msi} is for 32-bits and
\code{mpich2-1.4.1p1-win-x86-64.msi} is for 64-bits.
The installation is easily done with a few clicks. The
default environment and path are recommended.

For running MPI and \proglang{R}, users need to set \code{PATH} to the
\code{mpiexec.exe} and \code{Rscript.exe}. By default,
\begin{Command}
### Under command mode, or save in a batch file.
SET R_HOME=C:\Program Files\R\R-2.15.1
SET MPI_ROOT=C:\Program Files\MPICH2
SET PATH=%MPI_ROOT%\bin\;%R_HOME%\bin\;%PATH%
\end{Command}
is for the 64-bit MPICH2, but replace
\begin{Command}
SET MPI_ROOT=C:\Program Files (x86)\MPICH2
\end{Command}
for the 32-bit MPICH2.


\subsection[Install from Binary]{Install from Binary}
\label{sec:install_from_binary}
\addcontentsline{toc}{subsection}{\thesubsection. Install from Binary}

The binary packages of \pkg{pbdDEMO} are available on the website:
``Programming with Big Data in R'' at
\url{http://r-pbd.org/}.
Note that different MPI systems require different binaries.
The binary can be installed by
\begin{Command}
R CMD INSTALL pbdDEMO_0.1-0.zip
\end{Command}

As on Unix systems,
one can start quickly with \pkg{pbdDEMO} by learning from the
following demos. There are six basic examples.
\begin{Command}
### Run the demo with 2 processors by
mpiexec -np 2 Rscript.exe -e "demo(allgather,'pbdDEMO',ask=F,echo=F)"
mpiexec -np 2 Rscript.exe -e "demo(allreduce,'pbdDEMO',ask=F,echo=F)"
mpiexec -np 2 Rscript.exe -e "demo(bcast,'pbdDEMO',ask=F,echo=F)"
mpiexec -np 2 Rscript.exe -e "demo(gather,'pbdDEMO',ask=F,echo=F)"
mpiexec -np 2 Rscript.exe -e "demo(reduce,'pbdDEMO',ask=F,echo=F)"
mpiexec -np 2 Rscript.exe -e "demo(scatter,'pbdDEMO',ask=F,echo=F)"
\end{Command}
{\color{red}Warning:}
Note that spacing inside \code{demo} is not working for Windows systems
and \code{Rscript.exe} should be evoked rather than \code{Rscript}.


\subsection[Build from Source]{Build from Source}
\label{sec:building_from_source}
\addcontentsline{toc}{subsection}{\thesubsection. Build from Source}

{\color{red} \bf Warning:} This section is only for building binary in
32- and 64-bit Windows system. A more general way can be found in the file
\code{pbdDEMO/INSTALL.win}.

Make sure that \proglang{R}, \pkg{Rtools}, and \pkg{MINGW} are in the \code{PATH}.
See details on the website "Building R for Windows" at
\url{http://cran.r-project.org/bin/windows/Rtools/}.
But, if both 32- and 64-bits MPICH2 are installed, two different
environment variables \code{MPI_ROOT_32} and \code{MPI_ROOT_64}
need to be set for building binaries.

For example, the minimum requirement may be
\begin{Command}
### Under command mode, or save in a batch file.
SET R_HOME=C:\Program Files\R\R-2.15.1
SET RTOOLS=C:\Rtools\bin\
SET MINGW=C:\Rtools\gcc-4.6.3\bin
SET PATH=%R_HOME%;%R_HOME%\BIN\;%RTOOLS%;%MINGW%;%PATH%
SET MPI_ROOT_64=C:\Program Files\MPICH2
SET MPI_ROOT_32=C:\Program Files (x86)\MPICH2
\end{Command}

With a correct \code{PATH}, one can use the \proglang{R} commands
to install/build the \pkg{pbdDEMO}:
\begin{Command}
### Under command mode, build and install the binary.
tar zxvf pbdDEMO_0.1-0.tar.gz
R CMD INSTALL --build pbdDEMO
R CMD INSTALL pbdDEMO_0.1-0.zip
\end{Command}

