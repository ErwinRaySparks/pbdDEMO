
%%% ----------------------------------------------------------------------
%%%%%%%%%%%%%%%%%%%%%%%%%%%%%%%%%%%%%%%%%%%%%%%%%%%%%%%%%%%%%%%%%%%%%%%%%%


\chapter{MPI for the R User}
\label{chp:mpi_for_the_r_user}

\inspire{I cannot live without brain-work. What else is there to live for?}{Sherlock Holmes}
\vspace{0.5cm}


Cicero once said that ``If you have a garden and a library, you have
everything you need.'' So in that spirit, for the next two chapters we will
use the MPI library to get our hands dirty and root around in the dirt of
low-level MPI programming.

\section{MPI Basics}
\index{MPI}

In a sense, Cicero (in the above tortured metaphor) was quite right. MPI is
all that we \emph{need} in the same way that I might only \emph{need} bread
and cheese, but really what I \emph{want} is a pizza. MPI is somewhat
low-level and can be quite fiddly, but mastering it adds a very powerful
tool to the repertoire of the parallel \proglang{R} programmer, and is
essential for anyone who wants to do large scale development of parallel
codes.  

``MPI'' stands for ``Message Passing Interface''.\index{Message Passing 
Interface|see{MPI}} How it really works goes \emph{well} beyond the scope of 
this document. But at a basic level, the idea is that the user is running a 
code on different compute nodes that (usually) can not directly modify objects 
in each others' memory.  In order to have all of the nodes working together on 
a common problem, data and computation directives are passed around over the 
network (often over a specialized link called infiniband).  

At its core, MPI is a standard interface for managing communications
(data and instructions) between different nodes or computers.
There are several major implementations of this standard, and the one
you should use may depend on the machine you are using.
But this is a compiling issue, so user programs are unaffected beyond
this minor hurdle. Some of the most well-known implementations are
OpenMPI,\index{Library!OpenMPI} MPICH2,\index{Library!MPICH2} and
Cray MPT.\index{Library!MPT}

At the core of using MPI is the \emph{communicator}. At a technical level,
a communicator is a pretty complicated data structure, but these deep
technical details are not necessary for proceeding. We will instead think
of it somewhat like the post office. When we wish to send a letter
(communication) to someone else (another processor), we merely drop
the letter off at a post office mailbox (communicator) and trust that
the post office (MPI) will handle things accordingly (sort of).

The general process for directly --- or indirectly --- utilizing MPI in
SPMD\index{Parallelism!SPMD} programs goes something like this:

\begin{enumerate}
 \item Initialize communicator(s).
 \item Have each process read in its portion of the data.
 \item Perform computations.
 \item Communicate results.
 \item Shut down the communicator(s).
\end{enumerate}

Some of the above steps may be swept away under a layer of abstraction for
the user, but the need may arise where directly interfacing with MPI is not
only beneficial, but necessary.

More details and numerous examples using MPI with \proglang{R} are
available in the sections to follow, as well as in the \pkg{pbdMPI}
vignette.





\section{pbdMPI vs Rmpi}

There is another package on the CRAN which the \proglang{R} programmer may
use to interface with MPI, namely
\pkg{Rmpi}\index{Package!\pkg{Rmpi}}~\citep{Rmpi}. There are several issues
one must consider when choosing which package to use if one were to only
use one of them.
\begin{enumerate}
 \item $(+)$ \pkg{pbdMPI} is easier to install than \pkg{Rmpi}
 \item $(+)$ \pkg{pbdMPI} is easier to use than \pkg{Rmpi}
 \item\label{enum:perf} $(+)$ \pkg{pbdMPI} can often outperform \pkg{Rmpi}
 \item\label{enum:integrate} $(+)$ \pkg{pbdMPI} integrates with the rest of
       \proglang{pbdR}
 %
 \item\label{enum:dompi} $(-)$ \pkg{Rmpi} can be used with
      \pkg{foreach}~\citep{foreach}\index{Package!\pkg{foreach}} via
      \pkg{doMPI}~\citep{dompi}\index{Package!\pkg{doMPI}}
 \item $(-)$ \pkg{Rmpi} can be used in the manager/worker paradigm
\end{enumerate}

We do not believe that the above can be reduced to a zero-sum game with
unambiguous winner and loser. Ultimately the needs of the user (or
developer) are paramount. We believe that \proglang{pbdR}
makes a very good case for itself, especially in the SPMD style,
but it can not satisfy everyone. However, for the remainder of this
section, we will present the case for several of the, as yet,
unsubstantiated pluses above.

In the case of ease of use, \pkg{Rmpi} uses bindings very close to the
level as they are used in C's MPI API. Specifically, whenever performing,
for example, a reduction operation such as ``allreduce'', you must specify the
type of your data. For example, using \pkg{Rmpi}'s API
\begin{lstlisting}[language=rr]
mpi.allreduce(x, type = 1)
\end{lstlisting}
would perform the sum allreduce if the object \code{x} consists of integer
data, while
\begin{lstlisting}[language=rr]
mpi.allreduce(x, type = 2)
\end{lstlisting}
would be used if \code{x} consists of doubles. However, with \pkg{pbdMPI}
\begin{lstlisting}[language=rr]
allreduce(x)
\end{lstlisting}
is used for both by making use of \proglang{R}'s S4 system of object
oriented programming. This is not mere code
golfing\footnote{See \url{https://en.wikipedia.org/wiki/Code_golf}}
that we are engaging in. The concept of what ``type'' your data is
in \proglang{R} is fairly foreign to most \proglang{R} users, and
misusing the \code{type} argument in \pkg{Rmpi} is a very easy way
to crash your program. Even if you are more comfortable with statically
typed languages and have no problem with this concept, consider the
following example:

\begin{lstlisting}[language=rr,title=Types in R]
> is.integer(1)
[1] FALSE
> is.integer(2)
[1] FALSE
> is.integer(1:2)
[1] TRUE
\end{lstlisting}

There are good reasons for \proglang{R} Core to have made this choice;
that is not the point. The point is that because objects in
\proglang{R} are dynamically typed, having to know the type of your
data when utilizing \pkg{Rmpi} is a needless burden. Instead,
\pkg{pbdMPI} takes the approach of adding a small abstraction
layer on top (which we intend to show does not negatively impact
performance) so that the user need not worry about such fiddly details.

In terms of performance, \pkg{pbdMPI} can greatly outperform \pkg{Rmpi}.
We present here the results of a benchmark we performed comparing the
``allgather'' operation between the two packages~\citep{pdac}.
The benchmark consisted of calling the respective ``allgather''
function from each package on a randomly generated
$10,000 \times 10,000$ distributed matrix with entries coming from
the standard normal distribution, using different numbers of processors.
\begin{table}[h]
 \centering
 \caption[Benchmark Comparing \pkg{Rmpi} and \pkg{pbdMPI}]{Benchmark Comparing \pkg{Rmpi} and \pkg{pbdMPI}.  Run time in seconds is listed for each operation.  The speedup is relative to \pkg{Rmpi}.}
 \label{tab:allgather}
 \begin{tabular}{rrrr}\hline\hline
  Cores & \pkg{Rmpi} & \pkg{pbdMPI} & Speedup \\\hline
  32    & 24.6       & 6.7          & 3.67 \\
  64    & 25.2       & 7.1          & 3.55 \\
  128   & 22.3       & 7.2          & 3.10 \\
  256   & 22.4       & 7.1          & 3.15 \\\hline\hline
 \end{tabular}
\end{table}
Table~\ref{tab:allgather} shows the results for this test, and in each
case, \pkg{pbdMPI} is the clear victor.

Whichever package you choose, whichever your favorite, for the remainder
of this document we will be using (either implicitly or explicitly)
\pkg{pbdMPI}.  






\section{The GBD Data Structure}
\label{sec:gbdstruct}

This is the boring stuff everyone hates, but like your medicine, it's
ultimately better for you to just take it and get it out of the way:
data structures. In particular, we will be discussing a distributed
data structure that, for lack of a better name (and I assure you are
tried), we will call the GBD data structure.\index{GBD data structure}
This data structure is more paradigm or philosophy than a rigid data
structure like an array or list. Consider it a set of ``best practices'',
or if nothing else, a starting place if you have no idea how to proceed.

The GBD data structure is designed to fit the types of problems which
are arguably most common to data science, namely tall and skinny matrices.
It will work best with these (from a computational efficiency perspective)
problems, although that is not required. In fact, very little at all is
required of this data structure. At its core, the data structure is a
distributed matrix data structure, with the following rules:
  \begin{enumerate}
    \item \code{GBD} is \emph{distributed}. No one processor owns all of
          the matrix.
%     \item \code{GBD} is \emph{balanced}.  Every processor owns (roughly) the same amount of data.
    \item \code{GBD} is \emph{non-overlapping}. Any row owned by one
          processor is owned by no other processors.
    \item \code{GBD} is \emph{row-contiguous}. If a processor owns one
          element of a row, it owns the entire row.
    \item \code{GBD} is globally \emph{row-major}\footnote{In the sense of
          the data decomposition. More specifically, the global matrix is
          chopped up into local sub-matrices in a row-major way.},
          locally \emph{column-major}\footnote{The local sub-objects are
          \proglang{R} matrices, which are stored in column-major fashion.}.
          \index{GBD row-major matrix}\index{GBD column-major matrix}
    \item The last row of the local storage of a processor is adjacent
          (by global row) to the first row of the local storage of next
          processor (by communicator number) that owns data. That is, global
          row-adjacency is preserved in local storage via the
          communicator.\label{enum:confusing}
    \item \code{GBD} is (relatively) easy to understand, but can lead to
          bottlenecks if you have many more columns than rows.
  \end{enumerate}

Of this list, perhaps the most difficult to understand is
number~\ref{enum:confusing}. This is a precise, albeit cumbersome explanation
for a simple idea. If two processors are adjacent and each owns data,
then their local sub-matrices are adjacent row-wise as well. For example,
rows $n$ and $n+1$ of a matrix are adjacent; possible configurations for
the distributed ownership are processors $q$ owns row $n$ and $q+1$ owns
row $n+1$; processor $q$ owns row $n$, processor $q+1$ owns \emph{nothing},
and processor $q+2$ owns row $n+1$.

For some, no matter how much we try or what we write, the wall of text
simply will not suffice. So here are a few visual examples. Suppose we
have the global data matrix $x$, given as:

\begin{align*}
x &= \left[
      \begin{array}{lllllllll}
      x_{11} & x_{12} & x_{13} & x_{14} & x_{15} & x_{16} & x_{17} & x	_{18} & x_{19}\\
      x_{21} & x_{22} & x_{23} & x_{24} & x_{25} & x_{26} & x_{27} & x	_{28} & x_{29}\\
      x_{31} & x_{32} & x_{33} & x_{34} & x_{35} & x_{36} & x_{37} & x	_{38} & x_{39}\\
      x_{41} & x_{42} & x_{43} & x_{44} & x_{45} & x_{46} & x_{47} & x	_{48} & x_{49}\\
      x_{51} & x_{52} & x_{53} & x_{54} & x_{55} & x_{56} & x_{57} & x	_{58} & x_{59}\\
      x_{61} & x_{62} & x_{63} & x_{64} & x_{65} & x_{66} & x_{67} & x	_{68} & x_{69}\\
      x_{71} & x_{72} & x_{73} & x_{74} & x_{75} & x_{76} & x_{77} & x	_{78} & x_{79}\\
      x_{81} & x_{82} & x_{83} & x_{84} & x_{85} & x_{86} & x_{87} & x	_{88} & x_{89}\\
      x_{91} & x_{92} & x_{93} & x_{94} & x_{95} & x_{96} & x_{97} & x	_{98} & x_{99}
      \end{array}
\right]_{9\times 9}
\end{align*}
with processor array\footnote{Palette selected to be distinguishable by
the color blind, taken from
\url{http://jfly.iam.u-tokyo.ac.jp/color/\#pallet}}
(indexing always starts at 0 not 1)
\begin{align*}
\text{Processors = }
      \begin{array}{llllll}
      \color{g11}0 & \color{g12}1 & \color{g13}2 & \color{g21}3 & \color{g22}4 & \color{g23}5
      \end{array}
\end{align*}

Then we might split up and distribute the data onto processors like so:
\begin{align*}
x &= \left[
      \begin{array}{lllllllll}
      \color{g11}x_{11} & \color{g11}x_{12} & \color{g11}x_{13} & \color{g11}x_{14} & \color{g11}x_{15} & \color{g11}x_{16} & \color{g11}x_{17} & \color{g11}x_{18} & \color{g11}x_{19}\\
      %
      \color{g11}x_{21} & \color{g11}x_{22} & \color{g11}x_{23} & \color{g11}x_{24} & \color{g11}x_{25} & \color{g11}x_{26} & \color{g11}x_{27} & \color{g11}x_{28} & \color{g11}x_{29}\\\hline
      %
      \color{g12}x_{31} & \color{g12}x_{32} & \color{g12}x_{33} & \color{g12}x_{34} & \color{g12}x_{35} & \color{g12}x_{36} & \color{g12}x_{37} & \color{g12}x_{38} & \color{g12}x_{39}\\
      %
      \color{g12}x_{41} & \color{g12}x_{42} & \color{g12}x_{43} & \color{g12}x_{44} & \color{g12}x_{45} & \color{g12}x_{46} & \color{g12}x_{47} & \color{g12}x_{48} & \color{g12}x_{49}\\\hline
      %
      \color{g13}x_{51} & \color{g13}x_{52} & \color{g13}x_{53} & \color{g13}x_{54} & \color{g13}x_{55} & \color{g13}x_{56} & \color{g13}x_{57} & \color{g13}x_{58} & \color{g13}x_{59}\\
      %
      \color{g13}x_{61} & \color{g13}x_{62} & \color{g13}x_{63} & \color{g13}x_{64} & \color{g13}x_{65} & \color{g13}x_{66} & \color{g13}x_{67} & \color{g13}x_{68} & \color{g13}x_{69}\\\hline
      %
      \color{g21}x_{71} & \color{g21}x_{72} & \color{g21}x_{73} & \color{g21}x_{74} & \color{g21}x_{75} & \color{g21}x_{76} & \color{g21}x_{77} & \color{g21}x_{78} & \color{g21}x_{79}\\\hline
      %
      \color{g22}x_{81} & \color{g22}x_{82} & \color{g22}x_{83} & \color{g22}x_{84} & \color{g22}x_{85} & \color{g22}x_{86} & \color{g22}x_{87} & \color{g22}x_{88} & \color{g22}x_{89}\\\hline
      %
      \color{g23}x_{91} & \color{g23}x_{92} & \color{g23}x_{93} & \color{g23}x_{94} & \color{g23}x_{95} & \color{g23}x_{96} & \color{g23}x_{97} & \color{g23}x_{98} & \color{g23}x_{99}\\
      \end{array}
\right]_{9\times 9}
\end{align*}

With local storage view:
\begin{align*}
\left[\begin{array}{lllllllll}
      \color{g11}x_{11} & \color{g11}x_{12} & \color{g11}x_{13} & \color{g11}x_{14} & \color{g11}x_{15} & \color{g11}x_{16} & \color{g11}x_{17} & \color{g11}x_{18} & \color{g11}x_{19}\\
      \color{g11}x_{21} & \color{g11}x_{22} & \color{g11}x_{23} & \color{g11}x_{24} & \color{g11}x_{25} & \color{g11}x_{26} & \color{g11}x_{27} & \color{g11}x_{28} & \color{g11}x_{29}
\end{array}\right]_{2\times 9}
\\
\left[\begin{array}{lllllllll}
      \color{g12}x_{31} & \color{g12}x_{32} & \color{g12}x_{33} & \color{g12}x_{34} & \color{g12}x_{35} & \color{g12}x_{36} & \color{g12}x_{37} & \color{g12}x_{38} & \color{g12}x_{39}\\
      \color{g12}x_{41} & \color{g12}x_{42} & \color{g12}x_{43} & \color{g12}x_{44} & \color{g12}x_{45} & \color{g12}x_{46} & \color{g12}x_{47} & \color{g12}x_{48} & \color{g12}x_{49}
\end{array}\right]_{2\times 9}
\\
\left[\begin{array}{lllllllll}
      \color{g13}x_{51} & \color{g13}x_{52} & \color{g13}x_{53} & \color{g13}x_{54} & \color{g13}x_{55} & \color{g13}x_{56} & \color{g13}x_{57} & \color{g13}x_{58} & \color{g13}x_{59}\\
      \color{g13}x_{61} & \color{g13}x_{62} & \color{g13}x_{63} & \color{g13}x_{64} & \color{g13}x_{65} & \color{g13}x_{66} & \color{g13}x_{67} & \color{g13}x_{68} & \color{g13}x_{69}
\end{array}\right]_{2\times 9}
\\
\left[\begin{array}{lllllllll}
      \color{g21}x_{71} & \color{g21}x_{72} & \color{g21}x_{73} & \color{g21}x_{74} & \color{g21}x_{75} & \color{g21}x_{76} & \color{g21}x_{77} & \color{g21}x_{78} & \color{g21}x_{79}
\end{array}\right]_{1\times 9}
\\
\left[\begin{array}{lllllllll}
      \color{g22}x_{81} & \color{g22}x_{82} & \color{g22}x_{83} & \color{g22}x_{84} & \color{g22}x_{85} & \color{g22}x_{86} & \color{g22}x_{87} & \color{g22}x_{88} & \color{g22}x_{89}
\end{array}\right]_{1\times 9}
\\
\left[\begin{array}{lllllllll}
      \color{g23}x_{91} & \color{g23}x_{92} & \color{g23}x_{93} & \color{g23}x_{94} & \color{g23}x_{95} & \color{g23}x_{96} & \color{g23}x_{97} & \color{g23}x_{98} & \color{g23}x_{99}\\
\end{array}\right]_{1\times 9}
\end{align*}

This is a \emph{load balanced} approach, where we try to give each processor
roughly the same amount of stuff. Of course, that is not part of the
rules of the GBD structure, so we could just as well distribute
the data like so:
\begin{align*}
x &= \left[
      \begin{array}{lllllllll}
      \\\hline
      \color{g12}x_{11} & \color{g12}x_{12} & \color{g12}x_{13} & \color{g12}x_{14} & \color{g12}x_{15} & \color{g12}x_{16} & \color{g12}x_{17} & \color{g12}x_{18} & \color{g12}x_{19}\\
      %
      \color{g12}x_{21} & \color{g12}x_{22} & \color{g12}x_{23} & \color{g12}x_{24} & \color{g12}x_{25} & \color{g12}x_{26} & \color{g12}x_{27} & \color{g12}x_{28} & \color{g12}x_{29}\\
      %
      \color{g12}x_{31} & \color{g12}x_{32} & \color{g12}x_{33} & \color{g12}x_{34} & \color{g12}x_{35} & \color{g12}x_{36} & \color{g12}x_{37} & \color{g12}x_{38} & \color{g12}x_{39}\\
      %
      \color{g12}x_{41} & \color{g12}x_{42} & \color{g12}x_{43} & \color{g12}x_{44} & \color{g12}x_{45} & \color{g12}x_{46} & \color{g12}x_{47} & \color{g12}x_{48} & \color{g12}x_{49}\\\hline
      %%%%
      \color{g13}x_{51} & \color{g13}x_{52} & \color{g13}x_{53} & \color{g13}x_{54} & \color{g13}x_{55} & \color{g13}x_{56} & \color{g13}x_{57} & \color{g13}x_{58} & \color{g13}x_{59}\\
      %
      \color{g13}x_{61} & \color{g13}x_{62} & \color{g13}x_{63} & \color{g13}x_{64} & \color{g13}x_{65} & \color{g13}x_{66} & \color{g13}x_{67} & \color{g13}x_{68} & \color{g13}x_{69}\\\hline
      %%%%
      \color{g21}x_{71} & \color{g21}x_{72} & \color{g21}x_{73} & \color{g21}x_{74} & \color{g21}x_{75} & \color{g21}x_{76} & \color{g21}x_{77} & \color{g21}x_{78} & \color{g21}x_{79}\\\hline\\\hline
      %%%%
      \color{g23}x_{81} & \color{g23}x_{82} & \color{g23}x_{83} & \color{g23}x_{84} & \color{g23}x_{85} & \color{g23}x_{86} & \color{g23}x_{87} & \color{g23}x_{88} & \color{g23}x_{89}\\
      %
      \color{g23}x_{91} & \color{g23}x_{92} & \color{g23}x_{93} & \color{g23}x_{94} & \color{g23}x_{95} & \color{g23}x_{96} & \color{g23}x_{97} & \color{g23}x_{98} & \color{g23}x_{99}\\
      \end{array}
\right]_{9\times 9}
\end{align*}
With local storage view:
\begin{align*}
\left[\begin{array}{lllllllll}
      &&&&&&&&\hspace{4.87cm} 
\end{array}\right]_{0\times 9}
\\
\left[\begin{array}{lllllllll}
      \color{g12}x_{11} & \color{g12}x_{12} & \color{g12}x_{13} & \color{g12}x_{14} & \color{g12}x_{15} & \color{g12}x_{16} & \color{g12}x_{17} & \color{g12}x_{18} & \color{g12}x_{19}\\
      %
      \color{g12}x_{21} & \color{g12}x_{22} & \color{g12}x_{23} & \color{g12}x_{24} & \color{g12}x_{25} & \color{g12}x_{26} & \color{g12}x_{27} & \color{g12}x_{28} & \color{g12}x_{29}\\
      %
      \color{g12}x_{31} & \color{g12}x_{32} & \color{g12}x_{33} & \color{g12}x_{34} & \color{g12}x_{35} & \color{g12}x_{36} & \color{g12}x_{37} & \color{g12}x_{38} & \color{g12}x_{39}\\
      %
      \color{g12}x_{41} & \color{g12}x_{42} & \color{g12}x_{43} & \color{g12}x_{44} & \color{g12}x_{45} & \color{g12}x_{46} & \color{g12}x_{47} & \color{g12}x_{48} & \color{g12}x_{49}\\
\end{array}\right]_{4\times 9}
\\
\left[\begin{array}{lllllllll}
      \color{g13}x_{51} & \color{g13}x_{52} & \color{g13}x_{53} & \color{g13}x_{54} & \color{g13}x_{55} & \color{g13}x_{56} & \color{g13}x_{57} & \color{g13}x_{58} & \color{g13}x_{59}\\
      %
      \color{g13}x_{61} & \color{g13}x_{62} & \color{g13}x_{63} & \color{g13}x_{64} & \color{g13}x_{65} & \color{g13}x_{66} & \color{g13}x_{67} & \color{g13}x_{68} & \color{g13}x_{69}\\
\end{array}\right]_{2\times 9}
\\
\left[\begin{array}{lllllllll}
      \color{g21}x_{71} & \color{g21}x_{72} & \color{g21}x_{73} & \color{g21}x_{74} & \color{g21}x_{75} & \color{g21}x_{76} & \color{g21}x_{77} & \color{g21}x_{78} & \color{g21}x_{79}
\end{array}\right]_{1\times 9}
\\
\left[\begin{array}{lllllllll}
    &&&&&&&&\hspace{4.87cm} 
\end{array}\right]_{0\times 9}
\\
\left[\begin{array}{lllllllll}
      \color{g23}x_{81} & \color{g23}x_{82} & \color{g23}x_{83} & \color{g23}x_{84} & \color{g23}x_{85} & \color{g23}x_{86} & \color{g23}x_{87} & \color{g23}x_{88} & \color{g23}x_{89}\\
      \color{g23}x_{91} & \color{g23}x_{92} & \color{g23}x_{93} & \color{g23}x_{94} & \color{g23}x_{95} & \color{g23}x_{96} & \color{g23}x_{97} & \color{g23}x_{98} & \color{g23}x_{99}\\
\end{array}\right]_{2\times 9}
\end{align*}

Generally, we would recommend using a load balanced approach over this
bizarre distribution, although some problems may call for very strange data
distributions. For example, it is possible and common to have an empty matrix
after some subsetting or selectation.

With our first of two cumbersome data structures out of the way, we can
proceed to much more interesting content: actually using MPI.





\section{Common MPI Operations}

Fully explaining the process of MPI programming is a daunting task.
Thankfully, we can punt and merely highlight some key MPI operations and
how one should use them with \pkg{pbdMPI}.



\subsection{Basic Communicator Wrangling}

First things first, we must examine basic communicator issues, like
construction, destruction, and each processor's position within a communicator.

\begin{itemize}
  \item \textbf{Managing a Communicator}:  Create and destroy communicators.\\
        \code{init()} --- initialize communicator\\
        \code{finalize()} --- shut down communicator(s)
        \index{Code!\code{init()}}\index{Code!\code{finalize()}}
  %
  \item \textbf{Rank query}: Determine the processor's position in the
         communicator.\\
        \code{comm.rank()} --- ``who am I?''\\
        \code{comm.size()} --- ``how many of us are there?''
        \index{Code!\code{comm.rank()}}\index{Code!\code{comm.size()}}
  %
  \item \textbf{Barrier}: No processor can proceed until \emph{all} processors
        can proceed.\\
        \code{barrier()} --- ``computation wall'' that only all processors
        together can tear down.
        \index{Code!\code{barrier()}}
\end{itemize}


One quick word before proceeding. If a processor queries
\code{comm.size()}, this will return the total number of processors in
the communicators. However, communicator indexing is like indexing in
the programming language \proglang{C}. That is, the first element is
numbered 0 rather than 1. So when the first processor queries
\code{comm.rank()}, it will return 0, and when the last processor
queries \code{comm.rank()}, it will return \code{comm.size() - 1}.
  
We are finally ready to write our first MPI program:

\begin{lstlisting}[language=rr,title=Simple pbdMPI Example 1]
library(pbdMPI, quiet = TRUE)
init()

myRank <- comm.rank() + 1 # comm index starts at 0, not 1
print(myRank)

finalize()
\end{lstlisting}

Unfortunately, it is not very exciting, but you have to crawl before you
can drag race. Remember that all of our programs are written in SPMD style.
So this \emph{one} single program is written, and each processor will
execute the same program, but with different results, whence the name
``Single Program/Multiple Data''.\index{Parallelism!SPMD}

So what does it do? First we initialize the MPI communicator with the
call to \code{init()}. Next, we have each processor query its rank
via \code{comm.rank()}. Since indexing of MPI communicators starts at 0,
we add 1 because that is what we felt like doing. Finally we call
\proglang{R}'s \code{print()} function to print the result.
This printing is not particularly clever, and each processor will be
clamoring to dump its result to the output file/terminal. We will discuss
more sophisticated means of printing later. Finally, we shut down
the MPI communicator with \code{finalize()}.

If you were to save this program in the file \code{mpiex1.r} and you
wished to run it with 2 processors, you would issue the command:

\begin{Command}
### (Use Rscript.exe for windows system)
mpiexec -np 2 Rscript mpiex1.r
\end{Command}

To use more processors, you modify the \code{-np} argument
(``number processors'').  So if you want to use 4, you pass \code{-np 4}.

The above program technically, though not in spirit, bucks the trend of
officially opening with a ``Hello World'' program. So as not to incur
the wrath of the programming gods, we offer a simple such example by
slightly modifying the above program:

\begin{lstlisting}[language=rr,title=Simple pbdMPI Example 1.5]
library(pbdMPI, quiet = TRUE)
init()

myRank <- comm.rank()

if (myRank == 0){
  print("Hello, world.")
}

finalize()
\end{lstlisting}

One word of general warning we offer now is that we are taking very simple
approaches here for the sake of understanding, heavily relying on function
argument defaults. However, there are all kinds of crazy, needlessly
complicated things you can do with these functions.
See the \pkg{pbdMPI} reference manual for full details about how one may
utilize these (and other) \pkg{pbdMPI} functions.



\subsection{Reduce, Broadcast, and Gather}

Once managing a communicator is under control, you presumably want to do
things with all of your processors. The typical way you will have the
processors interact is given below:

\begin{itemize}
  \item \textbf{Reduction}: Say each processor has a number \code{x.gbd}.
        \index{Class!\code{.gbd}}
        Add all of them up, find the largest, find the smallest, \dots .\\
        \code{reduce(x.gbd, op='sum')}\index{Code!\code{reduce()}} ---
        only one processor gets result (default is 0)\\
        \code{allreduce(x.gbd, op='sum')}\index{Code!\code{allreduce()}} ---
        every processor gets result
  %
  \item \textbf{Gather}: Say each processor has a number.
        Create a new object on some processor(s) containing all of
        those numbers.\\
        \code{gather(x.gbd)}\index{Code!\code{gather()}} ---
        only one processor gets result\\
        \code{allgather(x.gbd)}\index{Code!\code{allgather()}} ---
          every processor gets result
  %
  \item \textbf{Broadcast}: One processor has a number \code{x.gbd} that
        every other processor should also have.\\
        \code{bcast(x.gbd)}\index{Code!\code{bcast()}}
\end{itemize}

Here perhaps explanations are inferior to examples; so without wasting any
more time, we proceed to the next example:

\begin{lstlisting}[language=rr,title=Simple pbdMPI Example 2]
library(pbdMPI, quiet = TRUE)
init()

n.gbd <- sample(1:10, size=1)

sm <- allreduce(n.gbd) # default op is 'sum'
print(sm)

gt <- allgather(n.gbd)
print(gt)

finalize()
\end{lstlisting}

So what does it do? First each processor samples a number from 1 to 10;
it is probably true that each processor will be using a different seed
for this sampling, though you should not rely on this alone to ensure
good parallel seeds. More on this particular problem in
Section~\ref{sec:pbdsugar} below.

Next, we perform an \code{allreduce()}\index{Code!\code{allreduce()}}
on \code{n.gbd}. Conceivably, the processors could have different
values for \code{n.gbd}.\index{Class!\.gbd} So the value of \code{n} is
local to each processor. So perhaps we want to add up all these numbers (with
as many numbers as there are processors) and store them in the global value
\code{sm} (for ``sum''). Each processor will agree as to the value
of \code{sm}, even if they disagree about the value of \code{n.gbd}.

Finally, we do the same but with an
\code{allgather()}\index{Code!\code{allgather()}} operation.

Try experimenting with this by running the program several times.
You should get different results each time. To make sure we have not
been lying to you about what is happening, you can even print the values
of \code{n.gbd} before the reduce and gather operations.



\subsection{Printing and RNG Seeds}\label{sec:pbdsugar}

In addition to the above common MPI operations, which will make up the bulk of
the MPI programmer's toolbox, we offer a few extra utility functions:

\begin{itemize}
  \item \textbf{Print}: printing with control over which processor prints.\\
        \code{comm.print(x, \dots)}\\
        \code{comm.cat(x, \dots)}
        \index{Code!\code{comm.print()}}\index{Code!\code{comm.cat()}}
  %
  \item \textbf{Random Seeds}: \\
        \code{comm.set.seed(seed, diff=FALSE)}: every processor uses the
        seed \code{seed}\index{Code!\code{comm.set.seed()}} \\
        \code{comm.set.seed(diff=TRUE)}: every processor uses an independent
        seed (via \pkg{rlecuyer})\index{Package!\pkg{rlecuyer}}
\end{itemize}

The \code{comm.print()} and \code{comm.cat()} functions are especially handy,
because they are much more sophisticated than their \proglang{R} counterparts
when using multiple processes. These functions which processes do the
printing, and if you choose to have all processes print their result, then
the printing occurs in an orderly fashion, with processor 0 getting the
first line, processor 1 getting the second, and so on.

For example, revisiting our ``Hello, world'' example, we can somewhat
simplify the code with a slight modification:

\begin{lstlisting}[language=rr,title=Simple pbdMPI Example 3]
library(pbdMPI, quiet = TRUE)
init()

myRank <- comm.rank()

comm.print("Hello, world.")

finalize()
\end{lstlisting}

If we want to see what each processor has to say, we can pass the optional
argument \code{all,rank=TRUE} to \code{comm.print()}. By default, each process
will print its rank, then what you told it to print. You can suppress the
printing of communicator rank via the optional argument \code{quiet=TRUE}
to \code{comm.print()}.

These functions are quite handy\dots
\begin{center}
\colorbox{yellow}{\Huge \color{red}{\#\#\#\#\# HOWEVER \#\#\#\#\# }}  
\end{center}
these functions are potentially dangerous, and so some degree of care
should be exercised. Indeed, it is possible to lock up all of the
active \proglang{R} sessions by incorrectly using them. Worse, achieving
this behavior is fairly easy to do. The way this occurs is by issuing
a \code{comm.print()} on an expression which requires communication.
For example, suppose we have a distributed object with local piece
\code{x.gbd} and a function \code{myFunction()} which requires
communication between the processors. Then calling
\begin{lstlisting}[language=rr,title=A Cautionary Tale of Printing in Parallel (1 of 3)]
print(myFunction(x.gbd)) 
\end{lstlisting}
is just fine, but will not have the nice orderly, behaved printing style
of \code{comm.print()}. However, if we issue
\begin{lstlisting}[language=rr,title=A Cautionary Tale of Printing in Parallel (2 of 3)]
comm.print(myFunction(x.gbd)) 
\end{lstlisting}
then we have just locked up all of the \proglang{R} processes. Indeed,
behind the scenes, a call somewhat akin to
\begin{lstlisting}[language=rr]
for (rank in 0:comm.size()){
  if (comm.rank() == rank){
    # do things
  }
  barrier()
}
\end{lstlisting}
has been ordered. The problem arises in the ``do things'' part. Since
(in our hypothetical example) the function \code{myFunction()} requires
communication between the processors, it will simply wait forever for
the others to respond until the job is killed. This is because the other
processors skipped over the ``do things'' part and are waiting at the barrier.
So lonely little processor 0 has been stood up, unable to communicate with
the remaining processors.

To avoid this problem, make it a personal habit to only print on
\emph{results}, not \emph{computations}. We can quickly rectify the above
example by doing the following:
\begin{lstlisting}[language=rr,title=A Cautionary Tale of Printing in Parallel (3 of 3)]
myResult <- myFunction(x.gbd)
comm.print(myResult)
\end{lstlisting}

In short, printing stored objects is safe. Printing a yet-to-be-evaluated
expression is not safe.





\subsection{Apply, Lapply, and Sapply}\label{sec:pbdsugar2}

But the \pkg{pbdMPI} sugar extends to more than just printing. We also have a
family of ``*ply'' functions, in the same vein as \proglang{R}'s
\code{apply()}, \code{lapply()}, and \code{sapply()}:
\begin{itemize}
  \item \textbf{Apply}: *ply-like functions.\\
  \code{pbdApply(X, MARGIN, FUN, \dots)} --- analogue of \code{apply()}\\
  \code{pbdLapply(X, FUN, \dots)} --- analogue of \code{lapply()}\\
  \code{pbdSapply(X, FUN, \dots)} --- analogue of \code{sapply()}
\end{itemize}
\index{Code!\code{pbdApply}}
\index{Code!\code{pbdLapply}}
\index{Code!\code{pbdSapply}}
For more efficient approach (non-barrier), one may consider use task pull
parallelism instead of ``*ply'' functions, see Section~\ref{sec:task_pull}
for more details.

Here is a simple example utilizing \code{pbdLapply()}:
\begin{lstlisting}[language=rr,title=Example 4]
library(pbdMPI, quiet = TRUE)
init()

n <- 100
x <- split((1:n) + n * comm.rank(), rep(1:10, each = 10))
sm <- pbdLapply(x, sum)
comm.print(unlist(sm))

finalize()
\end{lstlisting}

So what does it do? Why don't you tell us? We're busy people, after all,
and we're not going to be around forever. Try guessing what it will do,
then run the program to see if you are correct. As you evaluate this and
every parallel code, ask yourself which pieces involve communication and
which pieces are local computations.





\section{Timing MPI Tasks}

Measuring run time is a fundamental performance measure in computing.
However, in parallel computing, not all ``parallel components'' (e.g.
threads, or MPI processes) will take the same amount of time to complete
a task, even when all tasks are given completely identical jobs.
So measuring ``total run time'' begs the question, run time of what?

To help, we offer a timing function \code{demo.timer()} which can wrap
segments of code much in the same way that \code{system.time()} does.
However, the three numbers reported by \code{demo.timer()} are:
\begin{itemize}
\item the minimum elapsed time measured across all processes,
\item the average elapsed time measured across all processes, and
\item the maximum elapsed time across all processes.
\end{itemize}
The code for this function is listed below:

\begin{lstlisting}[language=rr,title=Timer Function]
demo.timer <- function(timed)
{
  ltime <- system.time(timed)[3]
  
  mintime <- allreduce(ltime, op='min')
  maxtime <- allreduce(ltime, op='max')
  
  meantime <- allreduce(ltime, op='sum') / comm.size()
  
  return( c(min=mintime, mean=meantime, max=maxtime) )
}
\end{lstlisting}



\section{Exercises}
\label{sec:mpi_for_the_r_user_exercise}

\begin{enumerate}[label=\thechapter-\arabic*]
\item Write a script that will have each processor randomly take a sample of
      size 1 of \code{TRUE} and \code{FALSE}. Have each processor print its
      result.\label{ex:mpi1}

\item 
Modify the script in Exercise~\ref{ex:mpi1} above to determine if any
processors sampled \code{TRUE}. Do the same to determine if all processors 
sampled \code{TRUE}. In each case, print the result. Compare to the functions 
\code{comm.all()} and \code{comm.any()}. {\color{blue} Hint:  use 
\code{allreduce()}.} 
\index{Code!\code{comm.all()}}\index{Code!\code{comm.any()}}

\item In \pkg{pbdMPI}, there is a parallel sorting function called      
\code{comm.sort()} which is similar to the usual \code{sort()} of 
\proglang{R}. Implement parallel equivalents to the usual \code{order()} 
and \code{rank()} of \proglang{R}.\label{ex:mpi2}\index{Code!\code{comm.sort()}}

\item 
Time the performance of Exercise~\ref{ex:mpi2}. Identify the need of MPI 
communications for different sorting or ordering algorithms.

\item 
There are ``parallel copycat'' versions of to R's *ply functions in several 
\proglang{R} packages, such as \code{mclapply()} (a parallel version of 
\code{lapply()}) in the \pkg{parallel} package.\index{Package!\pkg{parallel}}
Try to compare the difference and performance of those *ply-like functions.

\end{enumerate}



