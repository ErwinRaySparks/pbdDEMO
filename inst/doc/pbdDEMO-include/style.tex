%%%%%%%%%%%%%%%%%%%%%%%%%%%%%%%%%%%%%%%%%%%%%%%%%%%%%%%%%%%%%%%%%%%%%%%%%%%%%%%%%%%%%%%%%%%%%%%%%%%%%%%%%%%%%%%%%%%%%%%%%%%%%%
%%%%% definitions
%%%%%%%%%%%%%%%%%%%%%%%%%%%%%%%%%%%%%%%%%%%%%%%%%%%%%%%%%%%%%%%%%%%%%%%%%%%%%%%%%%%%%%%%%%%%%%%%%%%%%%%%%%%%%%%%%%%%%%%%%%%%%%

\makeatletter
\newcommand\code{\bgroup\@makeother\_\@makeother\~\@makeother\$\@codex}
\def\@codex#1{{\normalfont\ttfamily\hyphenchar\font=-1 #1}\egroup}
\makeatother
%%\let\code=\texttt
\let\proglang=\textsf

\newcommand{\pkg}[1]{{\fontseries{b}\selectfont #1}}

\newcommand{\bX}{\boldsymbol{X}}
\newcommand{\bx}{\boldsymbol{x}}
\newcommand{\by}{\boldsymbol{y}}
\newcommand{\bbeta}{\boldsymbol{\beta}}
\newcommand{\bepsilon}{\boldsymbol{\epsilon}}


\newcommand{\ltwo}[1]{\left|\left| #1 \right|\right|_2}
\newcommand{\norm}[1]{\left|\left| #1 \right|\right|}


\newcommand\mnote[1]{\marginpar{\vspace*{-.8cm}#1}}

\newcommand\easy{\mnote{\LARGE\color{green} \textbf{Easy}}}
\newcommand\medium{\mnote{\LARGE\color{yellow2} \textbf{Medium}}}
\newcommand\hard{\mnote{\LARGE\color{red} \textbf{Hard}}}

%%%%%%%%%%%%%%%%%%%%%%%%%%%%%%%%%%%%%%%%%%%%%%%%%%%%%%%%%%%%%%%%%%%%%%%%%%%%%%%%%%%%%%%%%%%%%%%%%%%%%%%%%%%%%%%%%%%%%%%%%%%%%%
%%%%% paragraph spacing
%%%%%%%%%%%%%%%%%%%%%%%%%%%%%%%%%%%%%%%%%%%%%%%%%%%%%%%%%%%%%%%%%%%%%%%%%%%%%%%%%%%%%%%%%%%%%%%%%%%%%%%%%%%%%%%%%%%%%%%%%%%%%%
\usepackage{parskip}
\setlength{\parskip}{.3cm}

%%%%%%%%%%%%%%%%%%%%%%%%%%%%%%%%%%%%%%%%%%%%%%%%%%%%%%%%%%%%%%%%%%%%%%%%%%%%%%%%%%%%%%%%%%%%%%%%%%%%%%%%%%%%%%%%%%%%%%%%%%%%%%
%%%%% colors
%%%%%%%%%%%%%%%%%%%%%%%%%%%%%%%%%%%%%%%%%%%%%%%%%%%%%%%%%%%%%%%%%%%%%%%%%%%%%%%%%%%%%%%%%%%%%%%%%%%%%%%%%%%%%%%%%%%%%%%%%%%%%%
\usepackage{xcolor}
\usepackage{color}

\definecolor{yellow2}{rgb}{1, .7, 0}

\definecolor{mygreen}{RGB}{0,150,0}

\definecolor{gray}{rgb}{.6,.6,.6}
\definecolor{dkgray}{rgb}{.3,.3,.3}
\definecolor{orange}{rgb}{1,0.5,0}
\definecolor{grayish}{rgb}{.9, .9, .9}

\definecolor{g11}{rgb}{0, 0, 1}
\definecolor{g12}{rgb}{0, .4, 1}
\definecolor{g13}{rgb}{0, .8, 1}

\definecolor{g21}{rgb}{0, .5, .3}
\definecolor{g22}{rgb}{.4, .5, .3}
\definecolor{g23}{rgb}{.8, .5, .3}

\definecolor{dkgreen}{rgb}{0,0.6,0}
\definecolor{mauve}{rgb}{0.58,0,0.82}

\definecolor{p0}{RGB}{150,0,0}
\definecolor{p1}{RGB}{0,150,0}
\definecolor{p2}{RGB}{0,0,150}
\definecolor{p3}{RGB}{150,75,0}


%%%%%%%%%%%%%%%%%%%%%%%%%%%%%%%%%%%%%%%%%%%%%%%%%%%%%%%%%%%%%%%%%%%%%%%%%%%%%%%%%%%%%%%%%%%%%%%%%%%%%%%%%%%%%%%%%%%%%%%%%%%%%%
%%%%% lstlisting
%%%%%%%%%%%%%%%%%%%%%%%%%%%%%%%%%%%%%%%%%%%%%%%%%%%%%%%%%%%%%%%%%%%%%%%%%%%%%%%%%%%%%%%%%%%%%%%%%%%%%%%%%%%%%%%%%%%%%%%%%%%%%%
\usepackage{listings}

\lstdefinelanguage{rr}{language=R, 
                      basicstyle=\ttfamily\color{black}, 
		      backgroundcolor=\color{grayish}, 
		      frame=single, 
		      breaklines=true, 
                      keywordstyle=\color{blue},
		      commentstyle=\color{dkgreen},
		      stringstyle=\color{mauve},
		      numbers=left,%none,
		      numberstyle=\tiny\color{dkgray},
		      stepnumber=1,       
		      numbersep=8pt,      
		      showspaces=false,      
		      showstringspaces=false,  
		      showtabs=false,     
		      rulecolor=\color{gray},   
		      tabsize=4,     
		      captionpos=t,
		      title=\lstname,
		      escapechar=?
% 		      escapeinside={(*@}{@*)},
% 		      escapebegin={\begin{lrbox}{0}\minipage[t]{.3\linewidth}\raggedleft\normalfont\itshape\small\leavevmode\color{black!70}\ignorespaces},
%                       escapeend={\endminipage\end{lrbox}\llap{\raisebox{0pt}[0pt][0pt]{\box0}\hspace{\comsep}}}
} 


\lstdefinelanguage{ft}{
  language=[90]Fortran,
  basicstyle=\ttfamily,
                      keywordstyle=\color{blue},
		      commentstyle=\color{dkgreen},
  morecomment=[l]{!\ }
}



\lstset{numbers=none,numberstyle=\footnotesize\ttfamily,
        frame=single,frameround=tttt,language=R,
        showspaces=false,showstringspaces=false,
        breaklines=true,breakatwhitespace=true,
        basicstyle=\small\ttfamily}
%\lstset{literate={<-}{{$\leftarrow$}}1}
\makeatletter
\let\Code\@undefined
\let\CodeInput\@undefined
\let\CodeOutput\@undefined
\makeatother
\lstnewenvironment{Command}[1][title=Shell Command]{\lstset{#1}}{}
\lstnewenvironment{Code}[1][title=R Script]{\lstset{#1}}{}
\lstnewenvironment{CodeOutput}[1][title=R Output]{\lstset{#1}}{}
\lstnewenvironment{Error}[1][]{
  \lstset{title=Error Message,basicstyle=\small\ttfamily}\color{Red}}{}
  
%%%%%%%%%%%%%%%%%%%%%%%%%%%%%%%%%%%%%%%%%%%%%%%%%%%%%%%%%%%%%%%%%%%%%%%%%%%%%%%%%%%%%%%%%%%%%%%%%%%%%%%%%%%%%%%%%%%%%%%%%%%%%%
%%%%% Chapter headings
%%%%%%%%%%%%%%%%%%%%%%%%%%%%%%%%%%%%%%%%%%%%%%%%%%%%%%%%%%%%%%%%%%%%%%%%%%%%%%%%%%%%%%%%%%%%%%%%%%%%%%%%%%%%%%%%%%%%%%%%%%%%%%
\usepackage[Bjornstrup]{fncychap}
% \usepackage[Lenny]{fncychap}


%%%%%%%%%%%%%%%%%%%%%%%%%%%%%%%%%%%%%%%%%%%%%%%%%%%%%%%%%%%%%%%%%%%%%%%%%%%%%%%%%%%%%%%%%%%%%%%%%%%%%%%%%%%%%%%%%%%%%%%%%%%%%%
%%%%% Headers/footers
%%%%%%%%%%%%%%%%%%%%%%%%%%%%%%%%%%%%%%%%%%%%%%%%%%%%%%%%%%%%%%%%%%%%%%%%%%%%%%%%%%%%%%%%%%%%%%%%%%%%%%%%%%%%%%%%%%%%%%%%%%%%%%

\usepackage{lastpage}
\usepackage{fancyhdr}

\pagestyle{fancy}

\newcommand{\prebodyheadfoot}{
  \fancyhf{} % clear all header and footer fields
  \fancyfoot{}
  \renewcommand{\headrulewidth}{0pt}
  \renewcommand{\footrulewidth}{0pt}
  
  % redefinition of the plain style:
  \fancypagestyle{plain}{%
  \fancyhf{} % clear all header and footer fields
  \renewcommand{\headrulewidth}{0pt}
  \renewcommand{\footrulewidth}{0pt}}
}

\newcommand{\bodyheadfoot}{
  \fancyhf{} % clear all header and footer fields
%   \fancyhead[L]{\slshape \rightmark}
  \fancyhead[L]{\slshape \leftmark}
  \fancyhead[R]{ \thepage\ of\ \pageref{LastPage}}
  \renewcommand{\headrulewidth}{1pt}
  \renewcommand{\footrulewidth}{0pt}
  
  % redefinition of the plain style:
  \fancypagestyle{plain}{%
  \fancyhf{} % clear all header and footer fields
  \renewcommand{\headrulewidth}{0pt}
  \renewcommand{\footrulewidth}{0pt}}
}

%%%%%%%%%%%%%%%%%%%%%%%%%%%%%%%%%%%%%%%%%%%%%%%%%%%%%%%%%%%%%%%%%%%%%%%%%%%%%%%%%%%%%%%%%%%%%%%%%%%%%%%%%%%%%%%%%%%%%%%%%%%%%%
%%%%% Fancy toc originally from http://tex.stackexchange.com/questions/35825/pretty-table-of-contents?lq=1
%%%%%%%%%%%%%%%%%%%%%%%%%%%%%%%%%%%%%%%%%%%%%%%%%%%%%%%%%%%%%%%%%%%%%%%%%%%%%%%%%%%%%%%%%%%%%%%%%%%%%%%%%%%%%%%%%%%%%%%%%%%%%%
\usepackage{xcolor}
\usepackage{framed}
\usepackage{tikz}
\usepackage{titletoc}
\usepackage{etoolbox}
\usepackage{lmodern}

% % definition of some personal colors
% \definecolor{myred}{RGB}{127,0,0}
% \definecolor{myyellow}{RGB}{169,121,69}
% 
% % command for the circle for the number of part entries
% \newcommand\Circle[1]{\tikz[overlay,remember picture] 
%   \node[draw,circle, text width=18pt,line width=1pt] {#1};}
% 
% % patching of \tableofcontents to use sans serif font for the tile
% \patchcmd{\tableofcontents}{\contentsname}{\sffamily\contentsname}{}{}
% % patching of \@part to typeset the part number inside a framed box in its own line
% % and to add color
% \makeatletter
% \patchcmd{\@part}
%   {\addcontentsline{toc}{part}{\thepart\hspace{1em}#1}}
%   {\addtocontents{toc}{\protect\addvspace{20pt}}
%     \addcontentsline{toc}{part}{\huge{\protect\color{myyellow}%
%       \setlength\fboxrule{2pt}\protect\Circle{%
%         \hfil\thepart\hfil%
%       }%
%     }\\[2ex]\color{myred}\sffamily#1}}{}{}
% 
% %\patchcmd{\@part}
% %  {\addcontentsline{toc}{part}{\thepart\hspace{1em}#1}}
% %  {\addtocontents{toc}{\protect\addvspace{20pt}}
% %    \addcontentsline{toc}{part}{\huge{\protect\color{myyellow}%
% %      \setlength\fboxrule{2pt}\protect\fbox{\protect\parbox[c][1em][c]{1.5em}{%
% %        \hfil\thepart\hfil%
% %      }}%
% %    }\\[2ex]\color{myred}\sffamily#1}}{}{}
% \makeatother
% 
% % this is the environment used to typeset the chapter entries in the ToC
% % it is a modification of the leftbar environment of the framed package
% \renewenvironment{leftbar}
%   {\def\FrameCommand{\hspace{6em}%
%     {\color{myyellow}\vrule width 2pt depth 6pt}\hspace{1em}}%
%     \MakeFramed{\parshape 1 0cm \dimexpr\textwidth-6em\relax\FrameRestore}\vskip2pt%
%   }
%  {\endMakeFramed}
% 
% % using titletoc we redefine the ToC entries for parts, chapters, sections, and subsections
% \titlecontents{part}
%   [0em]{\centering}
%   {\contentslabel}
%   {}{}
% \titlecontents{chapter}
%   [0em]{\vspace*{2\baselineskip}}
%   {\parbox{4.5em}{%
%     \hfill\Huge\sffamily\bfseries\color{myred}\thecontentspage}%
%    \vspace*{-2.3\baselineskip}\leftbar\textsc{\small\chaptername~\thecontentslabel}\\\sffamily}
%   {}{\endleftbar}
% \titlecontents{section}
%   [8.4em]
%   {\sffamily\contentslabel{3em}}{}{}
%   {\hspace{0.5em}\nobreak\itshape\color{myred}\contentspage}
% \titlecontents{subsection}
%   [8.4em]
%   {\sffamily\contentslabel{3em}}{}{}  
%   {\hspace{0.5em}\nobreak\itshape\color{myred}\contentspage}


%%%%%%%%%%%%%%%%%%%%%%%%%%%%%%%%%%%%%%%%%%%%%%%%%%%%%%%%%%%%%%%%%%%%%%%%%%%%%%%%%%%%%%%%%%%%%%%%%%%%%%%%%%%%%%%%%%%%%%%%%%%%%%
%%%%% misc.
%%%%%%%%%%%%%%%%%%%%%%%%%%%%%%%%%%%%%%%%%%%%%%%%%%%%%%%%%%%%%%%%%%%%%%%%%%%%%%%%%%%%%%%%%%%%%%%%%%%%%%%%%%%%%%%%%%%%%%%%%%%%%%

% \usepackage{url}
\usepackage{amsmath}
\usepackage{amsthm}
\usepackage{amssymb}
% \usepackage{subfig}


\usepackage{hyperref}

\hypersetup{
    pdfnewwindow=true,    
    colorlinks=true,      
    linkcolor=blue,      
    citecolor=blue,      
    filecolor=magenta,   
    urlcolor=blue      
}

\usepackage[absolute]{textpos}

%%%%%%%%%%%%%%%%%%%%%%%%%%%%%%%%%%%%%%%%%%%%%%%%%%%%%%%%%%%%%%%%%%%%%%%%%%%%%%%%%%%%%%%%%%%%%%%%%%%%%%%%%%%%%%%%%%%%%%%%%%%%%%
%%%%% Bibliography
%%%%%%%%%%%%%%%%%%%%%%%%%%%%%%%%%%%%%%%%%%%%%%%%%%%%%%%%%%%%%%%%%%%%%%%%%%%%%%%%%%%%%%%%%%%%%%%%%%%%%%%%%%%%%%%%%%%%%%%%%%%%%%

\usepackage[nottoc,numbib,notlof,notlot]{tocbibind}
\usepackage[authoryear,round]{natbib}
\usepackage{multirow}

\bibliographystyle{pbdDEMO-include/jss}

%%%%%%%%%%%%%%%%%%%%%%%%%%%%%%%%%%%%%%%%%%%%%%%%%%%%%%%%%%%%%%%%%%%%%%%%%%%%%%%%%%%%%%%%%%%%%%%%%%%%%%%%%%%%%%%%%%%%%%%%%%%%%%
%%%%% Exercise Itemize. 
%%%%%%%%%%%%%%%%%%%%%%%%%%%%%%%%%%%%%%%%%%%%%%%%%%%%%%%%%%%%%%%%%%%%%%%%%%%%%%%%%%%%%%%%%%%%%%%%%%%%%%%%%%%%%%%%%%%%%%%%%%%%%%
\usepackage{enumitem}


%%%%%%%%%%%%%%%%%%%%%%%%%%%%%%%%%%%%%%%%%%%%%%%%%%%%%%%%%%%%%%%%%%%%%%%%%%%%%%%%%%%%%%%%%%%%%%%%%%%%%%%%%%%%%%%%%%%%%%%%%%%%%%
%%%%% Index
%%%%%%%%%%%%%%%%%%%%%%%%%%%%%%%%%%%%%%%%%%%%%%%%%%%%%%%%%%%%%%%%%%%%%%%%%%%%%%%%%%%%%%%%%%%%%%%%%%%%%%%%%%%%%%%%%%%%%%%%%%%%%%

\usepackage{makeidx}
\makeindex
