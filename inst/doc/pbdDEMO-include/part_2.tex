\part{Distributed Matrix Methods}
\label{part:dmat}

The \pkg{pbdBASE} and \pkg{pbdDMAT} packages offer a distributed matrix class, \code{ddmatrix}, as well as a collection of high-level methods for performing common matrix operations.  For example, if you want to compute the mean of an \proglang{R} matrix \code{x}, you would call 
\begin{lstlisting}[language=rr]
mean(x)
\end{lstlisting}
That's exactly the same command you would issue if \code{x} is no longer an ordinary \proglang{R} matrix, but a distributed matrix.  These methods range from simple, embarrassingly parallel operations like sums and means, to tightly coupled linear algebra operations like matrix-matrix multiply and singular value decomposition.

Unfortunately, using these higher methods comes with a different cost:  getting the data into the distributed matrix class.  This can be especially frustrating because we assume that the any object of class \code{ddmatrix} is \emph{block cyclically distributed}.  This concept is discussed at length in the \pkg{pbdBASE} vignette \cite{Schmidt2012pbdBASEvignette}, and we do not intend to discuss the concept of a block cyclic data distribution at length herein.  However, we will demonstrate several examples of getting data into and out of the distributed block cyclic matrix format.

In short, once the hurdle of getting the data into the ``right format'' is out of the way, these methods offer very simple syntax (designed to mimic \proglang{R} as closely as possible) with the ability to scale computations on very large distributed machines.