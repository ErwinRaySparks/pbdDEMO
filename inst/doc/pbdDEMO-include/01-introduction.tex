
{\color{red} \bf Warning:} This document is written to explain the main
functions of \pkg{pbdDEMO}~\citep{Schmidt2012pbdDEMOpackage}, version 0.1-0.
Every effort will be made to ensure future versions are consistent with
these instructions, but features in later versions may not be explained
in this document.

Information about the functionality of this package,
and any changes in future versions can be found on website:
``Programming with Big Data in R'' at
\url{http://r-pbd.org/}.


\section[Introduction]{Introduction}
\label{sec:introduction}
\addcontentsline{toc}{section}{\thesection. Introduction}

This vignette is to explain some pbdR~\citep{pbdR2012}
examples which are higher level
applications and may be commonly found in basic Statistics.
The purposes is to show how to reuse the pre-exist functions, and
quickly solve problems in an efficient way.
The functions built for examples may not be exactly same idea of original
\proglang{R}~\citep{Rcore}
functions, but can be adjusted in similar wa.
You are very welcome to use these as templates and rewrite your own functions
or packages.


\subsection[Installation]{Installation}
\label{sec:installation}
\addcontentsline{toc}{subsection}{\thesubsection. Installation}

One can download \pkg{pbdDEMO} from CRAN at
\url{http://cran.r-project.org}, and
the intallation can be done with the following commands
\begin{Command}
tar zxvf pbdDEMO_0.1-0.tar.gz
R CMD INSTALL pbdDEMO
\end{Command}
Since \pkg{pbdEMO} depends on other \pkg{pbdR} packages,
please read the corresponding vignettes if installations did
not succeed.
We also provide several demos for the capability of \pkg{pbdR}
packages which are explained correspondingly in the next few sections.


\subsection[Notation]{Notation}
\label{sec:notation}
\addcontentsline{toc}{subsection}{\thesubsection. Notation}

We presume that readers already have idea about SPMD programming. If not,
please read \pkg{pbdMPI}'s vignette~\cite{Chen2012pbdMPIvignette} first.
If possible, readers are encouraged to run the demo of \pkg{pbdMPI} package
and go through the code step by step.

Note that we tend to use suffix \code{.spmd} to indicate a distributed
local object which is a portion of big object.
As the usual \proglang{R},
Supposer $\bX$ is a very big matrix, then
\code{X.spmd}'s could be R objects distributed on all processors,
and usually they have the same data type.
$\bX$ could be a result of \code{cbind} or \code{rbind} on all
\code{X.spmd}'s.
Examples can be found in the Section~\ref{sec:statistics_examples}.

We also tend to assume common variables without suffix \code{.spmd}
in SPMD programming.
For instance, \code{y} could be an R object on all processors.
Their dimension/attributes are allowed to be differed on some processors.
In SPMD case, it may be a good idea to invent a \code{spmd} S4 class
or methods for this purpose, but most S4 methods are sufficient
and fast in most computation.

While in distributed computing,
in contrast, it is a better idea to invent a \code{ddmatrix} class as in
\pkg{pbdBASE}~\citep{Schmidt2012pbdBASEpackage} and
\pkg{pbdDMAT}~\citep{Schmidt2012pbdDMATpackage}, then let efficient libraries
handle computing and avoid tedious coding.
i.e. $\bX$ is a very big matrix, but \code{X.dmat} is
in block-cyclic format where \code{X.dmat} can be utilized by most
\pkg{pbdR} functions.
Examples can be found in the Section~\ref{sec:data_input_examples}.

