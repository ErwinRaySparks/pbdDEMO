
{\color{red} \bf Warning:} This document is written to explain the main
functions of \pkg{pbdDEMO}~\citep{Schmidt2012pbdDEMOpackage}, version 0.1-0.
Every effort will be made to ensure future versions are consistent with
these instructions, but features in later versions may not be explained
in this document.

Information about the functionality of this package,
and any changes in future versions can be found on website:
``Programming with Big Data in R'' at
\url{http://r-pbd.org/}.


\section[Introduction]{Introduction}
\label{sec:introduction}
\addcontentsline{toc}{section}{\thesection. Introduction}

This vignette is to explain some pbdR~\citep{pbdR2012}
examples which are higher level
applications and may be commonly found in basic Statistics.
The purposes is to show how to reuse the pre-exist functions, and
quickly solve problems in an efficient way.
The functions built for examples may not be exactly same idea of original
\proglang{R}~\citep{Rcore}
functions, but can be adjusted in similar wa.
You are very welcome to use these as templates and rewrite your own functions
or packages.


\subsection[Installation and Quick Start]{Installation and Quick Start}
\label{sec:installation}
\addcontentsline{toc}{subsection}{\thesubsection. Installation and Quick Start}

One can download \pkg{pbdDEMO} from CRAN at
\url{http://cran.r-project.org}, and
the intallation can be done with the following commands
\begin{Command}
tar zxvf pbdDEMO_0.1-0.tar.gz
R CMD INSTALL pbdDEMO
\end{Command}

