{\color{red} \bf Warning:} This document is written to explain the main
functions of \pkg{pbdDEMO}~\citep{Schmidt2012pbdDEMOpackage}, version 0.1-0.
Every effort will be made to ensure future versions are consistent with
these instructions, but features in later versions may not be explained
in this document.

Information about the functionality of this package,
and any changes in future versions can be found on website:
``Programming with Big Data in R'' at
\url{http://r-pbd.org/}.


\part{Preliminaries}


\section[Introduction]{Introduction}
\label{sec:introduction}
\addcontentsline{toc}{section}{\thesection. Introduction}

This vignette is to explain some pbdR~\citep{pbdR2012}
examples which are higher level
applications and may be commonly found in basic Statistics.
The purposes is to show how to reuse the pre-exist functions, and
quickly solve problems in an efficient way.
The functions built for examples may not be exactly same idea of original
\proglang{R}~\citep{Rcore}
functions, but can be adjusted in similar wa.
You are very welcome to use these as templates and rewrite your own functions
or packages.

We presume that the reader already has an idea about SPMD programming. If not,
please read \pkg{pbdMPI}'s vignette~\cite{Chen2012pbdMPIvignette} first.
Ideally, readers should run the demos of the \pkg{pbdMPI} package,
and go through the code step by step.

\subsection[Installation]{Installation}
\label{sec:installation}
\addcontentsline{toc}{subsection}{\thesubsection. Installation}

One can download \pkg{pbdDEMO} from CRAN at
\url{http://cran.r-project.org}, and
the intallation can be done with the following commands
\begin{Command}
tar zxvf pbdDEMO_0.1-0.tar.gz
R CMD INSTALL pbdDEMO
\end{Command}
Since \pkg{pbdEMO} depends on other \pkg{pbdR} packages,
please read the corresponding vignettes if installations did
not succeed.
We also provide several demos for the capability of \pkg{pbdR}
packages which are explained correspondingly in the next few sections.


\subsection{List of Demos}
\addcontentsline{toc}{subsection}{\thesubsection. List of Demos}

\begin{lstlisting}[title=Shell Script]
### Under command mode, run the demo with 4 processors by
### (Use Rscript.exe for windows system)

# ----------------- #
# II Ad Hoc Methods #
# ----------------- #

# Monte carlo simulation
mpiexec -np 4 Rscript -e "demo(monte_carlo, package='pbdDMAT',ask=F,echo=F)"
# Sample mean and variance
mpiexec -np 4 Rscript -e "demo(sample_stat, package='pbdDMAT',ask=F,echo=F)"
# Binning
mpiexec -np 4 Rscript -e "demo(binning, package='pbdDMAT',ask=F,echo=F)"
# Quantile
mpiexec -np 4 Rscript -e "demo(quantile, package='pbdDMAT',ask=F,echo=F)"
# OLS
mpiexec -np 4 Rscript -e "demo(ols, package='pbdDMAT',ask=F,echo=F)"

# -------------------- #
# III ddmatrix Methods #
# -------------------- #

# Random matrix generation
mpiexec -np 4 Rscript -e "demo(randmat_global, package='pbdDMAT',ask=F,echo=F)"
mpiexec -np 4 Rscript -e "demo(randmat_local, package='pbdDMAT',ask=F,echo=F)"
# SPMD to DMAT
mpiexec -np 4 Rscript -e "demo(spmd_dmat, package='pbdDMAT',ask=F,echo=F)"
# Parallel CSV read
mpiexec -np 4 Rscript -e "demo(read_csv, package='pbdDMAT',ask=F,echo=F)"
# Parallel SQL read
mpiexec -np 4 Rscript -e "demo(read_sql, package='pbdDMAT',ask=F,echo=F)"
\end{lstlisting}













\section{Background}
\label{sec:bg}
\addcontentsline{toc}{section}{\thesection. Background}



\subsection{Terminology}
\addcontentsline{toc}{subsection}{\thesubsection. Terminology}

OOP stuff

\subsection[Notation]{Notation}
\label{sec:notation}
\addcontentsline{toc}{subsection}{\thesubsection. Notation}

Note that we tend to use suffix \code{.spmd} for an object when we wish to indicate that the object is distributed.  This is purely for pedagogical convenience, and has no semantic meaning.  Since the code is written in SPMD style, you can think of such objects as referring to either a large, global object, or to a processor's local piece of the whole (depending on context).  This is less confusing that it might first sound.

%% What?
% As the usual \proglang{R},
% Supposer $\bX$ is a very big matrix, then
% \code{X.spmd}'s could be R objects distributed on all processors,
% and usually they have the same data type.
% $\bX$ could be a result of \code{cbind} or \code{rbind} on all
% \code{X.spmd}'s.
% Examples can be found in the Section~\ref{sec:statistics_examples}.

We will not use this suffix to denote a global object common to all processors.  As a simple example, you could imagine having a large matrix with (global) dimensions $m\times n$ with each processor owning different collections of rows of the matrix.  All processors might need to know the values for $m$ and $n$; however, $m$ and $n$ do not depend on the local process, and so these do not receive the \code{.spmd} suffix.  In many cases, it may be a good idea to invent an S4 class object and a corresponding set of methods.  Doing so can greatly improve the usability and readability of your code, but is never strictly necessary.  However, these constructions are the foundation of the \pkg{pbdBASE}~\citep{Schmidt2012pbdBASEpackage} and
\pkg{pbdDMAT}~\citep{Schmidt2012pbdDMATpackage} packages.

On that note, depending on your requirements in distributed computing with \proglang{R}, it may be beneficial to you to use higher pbdR toolchain.  If you need to perform dense matrix operations, or statistical analyses which depend heavily on matrix algebra (linear modeling, principal components analysis, \dots), then the \pkg{pbdBASE} and \pkg{pbdDMAT} packages are a natural choice.  The major hurdle to using these tools is getting the data into the appropriate \code{ddmatrix} format, although we provide many tools to ease the pains of doing so.  Learning how to use these packages can greatly improve code performance, and take your statistical modeling in \proglang{R} to previously unimaginable scales.

Again for the sake of understanding, we will at times append the suffix \code{.dmat} to objects of class \code{ddmatrix}.  As with \code{.spmd}, this carries no semantic meaning, and is merely used to improve the readability of example code (especially when managing both ``\code{.spmd}'' and \code{ddmatrix} objects).


\subsection{Timing Jobs}
\addcontentsline{toc}{subsection}{\thesubsection. Timing Jobs}

Measuring run time is a fundamental performance measure in computing.  However, in parallel computing, not all ``parallel components'' (e.g. threads, or MPI processes) will take the same amount of time to complete a task, even when all tasks are given completely identical jobs.  So measuring ``total run time'' begs the question, run time of what?

To help, we offer a timing function \code{demo.timer()} which can wrap segments of code much in the same way that \code{system.time()} does.  However, the three numbers reported by \code{demo.timer()} are: (1) the minimum elapsed time measured across all processes, (2) the average elapsed time measured across all processes, and (3) the maximum elapsed time across all processes.  The code for this function is listed below:

\begin{lstlisting}[language=rr,title=Timer Function]
demo.timer <- function(timed)
{
  ltime <- system.time(timed)[3]
  barrier()
  
  mintime <- allreduce(ltime, op='min')
  maxtime <- allreduce(ltime, op='max')
  
  meantime <- allreduce(ltime, op='sum') / comm.size()
  
  return( c(min=mintime, mean=meantime, max=maxtime) )
}
\end{lstlisting}

