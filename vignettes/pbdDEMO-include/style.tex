%%%%%%%%%%%%%%%%%%%%%%%%%%%%%%%%%%%%%%%%%%%%%%%%%%%%%%%%%%%%%%
%%%%% definitions
%%%%%%%%%%%%%%%%%%%%%%%%%%%%%%%%%%%%%%%%%%%%%%%%%%%%%%%%%%%%%%


\newcommand{\bA}{\boldsymbol{A}}
\newcommand{\bB}{\boldsymbol{B}}
\newcommand{\bD}{\boldsymbol{D}}
\newcommand{\bI}{\boldsymbol{I}}
\newcommand{\bQ}{\boldsymbol{Q}}
\newcommand{\bU}{\boldsymbol{U}}
\newcommand{\bX}{\boldsymbol{X}}
\newcommand{\bx}{\boldsymbol{x}}
\newcommand{\by}{\boldsymbol{y}}
\newcommand{\bbeta}{\boldsymbol{\beta}}
\newcommand{\bepsilon}{\boldsymbol{\epsilon}}
\newcommand{\boldeta}{\boldsymbol{\eta}}
\newcommand{\bTheta}{\boldsymbol{\Theta}}
\newcommand{\btheta}{\boldsymbol{\theta}}
\newcommand{\bmu}{\boldsymbol{\mu}}
\newcommand{\bSigma}{\boldsymbol{\Sigma}}
\newcommand{\bsigma}{\boldsymbol{\sigma}}
\newcommand{\argmax}{\operatornamewithlimits{argmax}}
\newcommand{\Prob}{\mathbb{P}}
\newcommand{\E}{\mathbb{E}}
\newcommand{\mF}{\mathcal F}
\newcommand{\mS}{\mathcal{S}}
\newcommand{\bone}{\boldsymbol{1}}
\newcommand{\bzero}{\boldsymbol{0}}
\newcommand{\blambda}{\boldsymbol{\lambda}}


\newcommand{\ltwo}[1]{\left|\left| #1 \right|\right|_2}
\newcommand{\norm}[1]{\left|\left| #1 \right|\right|}


%%%%%%%%%%%%%%%%%%%%%%%%%%%%%%%%%%%%%%%%%%%%%%%%%%%%%%%%%%%%%%
%%%%% paragraph spacing
%%%%%%%%%%%%%%%%%%%%%%%%%%%%%%%%%%%%%%%%%%%%%%%%%%%%%%%%%%%%%%
\usepackage{parskip}
\setlength{\parskip}{.3cm}

%%%%%%%%%%%%%%%%%%%%%%%%%%%%%%%%%%%%%%%%%%%%%%%%%%%%%%%%%%%%%%
%%%%% colors
%%%%%%%%%%%%%%%%%%%%%%%%%%%%%%%%%%%%%%%%%%%%%%%%%%%%%%%%%%%%%%
\usepackage{xcolor}
\usepackage{color}

\definecolor{yellow2}{rgb}{1, .7, 0}

\definecolor{mygreen}{RGB}{0,150,0}

\definecolor{gray}{rgb}{.6,.6,.6}
\definecolor{dkgray}{rgb}{.3,.3,.3}
\definecolor{orange}{rgb}{1,0.5,0}
\definecolor{grayish}{rgb}{.9, .9, .9}

% % \definecolor{g11}{rgb}{0, 0, 1}
% % \definecolor{g12}{rgb}{0, .4, 1}
% % \definecolor{g13}{rgb}{0, .8, 1}
% % 
% % \definecolor{g21}{rgb}{0, .5, .3}
% % \definecolor{g22}{rgb}{.4, .5, .3}
% % \definecolor{g23}{rgb}{.8, .5, .3}

% changed for the colorblind, taken from http://jfly.iam.u-tokyo.ac.jp/color/#pallet
\definecolor{g11}{rgb}{.90, .60, 0}
\definecolor{g12}{rgb}{.35, .70, .90}
\definecolor{g13}{rgb}{0, .60, .50}

\definecolor{g21}{rgb}{.80, .40, 0}
\definecolor{g22}{rgb}{0, .45, .70}
\definecolor{g23}{rgb}{.80, .60, .70}

\definecolor{dkgreen}{rgb}{0,0.6,0}
\definecolor{mauve}{rgb}{0.58,0,0.82}

\definecolor{p0}{RGB}{150,0,0}
\definecolor{p1}{RGB}{0,150,0}
\definecolor{p2}{RGB}{0,0,150}
\definecolor{p3}{RGB}{150,75,0}


\definecolor{mygreen3}{rgb}{0,0.69,0.09}
\definecolor{mymagenta3}{rgb}{0.8,0,0.8}
\definecolor{myblue3}{rgb}{0,0,0.8}
\definecolor{myred}{rgb}{1,0,0}
\newcommand{\colorA}{\mbox{\color{mygreen3} A}}
\newcommand{\colorC}{\mbox{\color{mymagenta3} C}}
\newcommand{\colorG}{\mbox{\color{myblue3} G}}
\newcommand{\colorT}{\mbox{\color{myred} T}}


%%%%%%%%%%%%%%%%%%%%%%%%%%%%%%%%%%%%%%%%%%%%%%%%%%%%%%%%%%%%%%
%%%%% lstlisting
%%%%%%%%%%%%%%%%%%%%%%%%%%%%%%%%%%%%%%%%%%%%%%%%%%%%%%%%%%%%%%
\usepackage{listings}

\lstdefinelanguage{rr}{language=R, 
                      basicstyle=\ttfamily\color{black}, 
		      backgroundcolor=\color{grayish}, 
		      frame=single, 
		      breaklines=true, 
                      keywordstyle=\color{blue},
		      commentstyle=\color{dkgreen},
		      stringstyle=\color{mauve},
		      numbers=left,%none,
		      numberstyle=\tiny\color{dkgray},
		      stepnumber=1,       
		      numbersep=8pt,      
		      showspaces=false,      
		      showstringspaces=false,  
		      showtabs=false,     
		      rulecolor=\color{gray},   
		      tabsize=4,     
		      captionpos=t,
		      title=\lstname,
		      escapechar=?
% 		      escapeinside={(*@}{@*)},
% 		      escapebegin={\begin{lrbox}{0}\minipage[t]{.3\linewidth}\raggedleft\normalfont\itshape\small\leavevmode\color{black!70}\ignorespaces},
%                       escapeend={\endminipage\end{lrbox}\llap{\raisebox{0pt}[0pt][0pt]{\box0}\hspace{\comsep}}}
} 


\lstdefinelanguage{ft}{
  language=[90]Fortran,
  basicstyle=\ttfamily,
                      keywordstyle=\color{blue},
		      commentstyle=\color{dkgreen},
  morecomment=[l]{!\ }
}



\lstset{numbers=none,numberstyle=\footnotesize\ttfamily,
        frame=single,frameround=tttt,language=R,
        showspaces=false,showstringspaces=false,
        breaklines=true,breakatwhitespace=true,
        basicstyle=\small\ttfamily}
%\lstset{literate={<-}{{$\leftarrow$}}1}

  
%%%%%%%%%%%%%%%%%%%%%%%%%%%%%%%%%%%%%%%%%%%%%%%%%%%%%%%%%%%%%%
%%%%% pseudocode
%%%%%%%%%%%%%%%%%%%%%%%%%%%%%%%%%%%%%%%%%%%%%%%%%%%%%%%%%%%%%%

\usepackage{algpseudocode}

\newcommand{\pseudocode}[1]{
  {\centering Pseudocode\\[.1cm]}
  \fbox{\noindent
  \begin{minipage}{\dimexpr\textwidth-2\fboxsep-2\fboxrule\relax}
  \begin{algorithmic}[1]
  #1
  \end{algorithmic}
  \end{minipage}
  }\\
}

%%%%%%%%%%%%%%%%%%%%%%%%%%%%%%%%%%%%%%%%%%%%%%%%%%%%%%%%%%%%%%
%%%%% Chapter headings
%%%%%%%%%%%%%%%%%%%%%%%%%%%%%%%%%%%%%%%%%%%%%%%%%%%%%%%%%%%%%%
\usepackage[Bjornstrup]{fncychap}
% \usepackage[Lenny]{fncychap}


%%%%%%%%%%%%%%%%%%%%%%%%%%%%%%%%%%%%%%%%%%%%%%%%%%%%%%%%%%%%%%
%%%%% Headers/footers
%%%%%%%%%%%%%%%%%%%%%%%%%%%%%%%%%%%%%%%%%%%%%%%%%%%%%%%%%%%%%%

\usepackage{lastpage}
\usepackage{fancyhdr}

\pagestyle{fancy}

%%%%%%%%%%%%%%%%%%%%%%%%%%%%%%%%%%%%%%%%%%%%%%%%%%%%%%%%%%%%%%%%%%%%%%%%%%%%%%%%%%%%%%%%%%%%%%%%%%%%
%%%%%%%%%%%%%%%%%%%%%%%%%%
%%%%% Fancy toc originally from http://tex.stackexchange.com/questions/35825/pretty-table-of-contents?lq=1
%%%%%%%%%%%%%%%%%%%%%%%%%%%%%%%%%%%%%%%%%%%%%%%%%%%%%%%%%%%%%%
\usepackage{xcolor}
\usepackage{framed}
\usepackage{tikz}
\usepackage{titletoc}
\usepackage{etoolbox}
\usepackage{lmodern}


%%%%%%%%%%%%%%%%%%%%%%%%%%%%%%%%%%%%%%%%%%%%%%%%%%%%%%%%%%%%%%
%%%%% misc.
%%%%%%%%%%%%%%%%%%%%%%%%%%%%%%%%%%%%%%%%%%%%%%%%%%%%%%%%%%%%%%

% \usepackage{url}
\usepackage{amsmath}
\usepackage{amsthm}
\usepackage{amssymb}

\usepackage{caption}
\usepackage{subcaption}


\usepackage{hyperref}

\definecolor{Red}{rgb}{0.5,0,0}
\definecolor{Blue}{rgb}{0,0,0.5}
\hypersetup{
    pdfnewwindow=true,    
    colorlinks=true,      
    linkcolor=blue,      
    citecolor=blue,      
    filecolor=magenta,   
    linkcolor=blue,
    citecolor=Blue,
    urlcolor=Blue
}

\usepackage[absolute]{textpos}

%%% quotes
\newcommand\quelle[1]{{%
      \unskip\nobreak\hfil\penalty50
      \hskip2em\hbox{}\nobreak\hfil\textbf{#1}%
      \parfillskip=0pt \finalhyphendemerits=0 \par}}
% \newcommand{\inspire}[2]{{\raggedleft{\textit{#1}}\\{{\hfill---#2}}}}
\newcommand{\inspire}[2]{{\hfill\begin{minipage}[r]{.5\textwidth}\textit{#1}\\\quelle{---#2}\end{minipage}}}

%%%%%%%%%%%%%%%%%%%%%%%%%%%%%%%%%%%%%%%%%%%%%%%%%%%%%%%%%%%%%%
%%%%% Bibliography
%%%%%%%%%%%%%%%%%%%%%%%%%%%%%%%%%%%%%%%%%%%%%%%%%%%%%%%%%%%%%%

\usepackage[nottoc,numbib,notlof,notlot]{tocbibind}
\usepackage[authoryear,round]{natbib}
\usepackage{multirow}

\bibliographystyle{pbdDEMO-include/jss}

%%%%%%%%%%%%%%%%%%%%%%%%%%%%%%%%%%%%%%%%%%%%%%%%%%%%%%%%%%%%%%
%%%%% Exercise Itemize. 
%%%%%%%%%%%%%%%%%%%%%%%%%%%%%%%%%%%%%%%%%%%%%%%%%%%%%%%%%%%%%%
\usepackage{enumitem}


%%%%%%%%%%%%%%%%%%%%%%%%%%%%%%%%%%%%%%%%%%%%%%%%%%%%%%%%%%%%%%
%%%%% Index
%%%%%%%%%%%%%%%%%%%%%%%%%%%%%%%%%%%%%%%%%%%%%%%%%%%%%%%%%%%%%%

\usepackage{makeidx}
\makeindex



%%%%%%%%%%%%%%%%%%%%%%%%%%%%%%%%%%%%%%%%%%%%%%%%%%%%%%%%%%%%%%
%%%%% Chapter boxes
%%%%%%%%%%%%%%%%%%%%%%%%%%%%%%%%%%%%%%%%%%%%%%%%%%%%%%%%%%%%%%
\usepackage{rotating}
\usepackage[absolute]{textpos}
\usepackage{everypage}
\usepackage{nameref}

%declaration of the box that will contain the colored frame and its label
\newsavebox\chapbox

%creation of the colored frame
\newcommand\mylabel[1]{%
  \savebox\chapbox{\colorbox{grayish}{\parbox[c][28cm][t]{1cm}{\centering\textcolor{black}{\large #1}}}}}

\AddEverypageHook{%
    \begin{textblock*}{2.4cm}(20.4cm,0cm)
      \usebox\chapbox
    \end{textblock*}
} 


%%%%%%%%%%%%%%
%%%%% pbd
%%%%%%%%%%%%%%%

\usepackage{xspace}

\definecolor{pbdgrn}{HTML}{005700}
\definecolor{pbdrd}{HTML}{ab0000}
\definecolor{pbdylw}{HTML}{ab7e00}
\definecolor{pbdblu}{HTML}{2b74ec}
\newcommand{\pbdR}{%
\textbf{\color{pbdgrn}{p}\color{pbdrd}{b}\color{pbdylw}{d}\color{pbdblu}{R}}%
\xspace}
 
 
\makeatletter
\newcommand\packageversion[1]{\renewcommand\@packageversion{#1}}
\newcommand\@packageversion{}
\newcommand\demoversion{\@packageversion}
\makeatother
